%%% THIS FILE IS PLAIN TEX, run
%%  > tex Summary; dvips -Ppdf -o Summary.ps Summary.dvi; ps2pdf14 Summary.ps Summary.pdf

\voffset +0.4 true in
\hsize 6.5 true in
\vsize 9.0 true in
%\font\typo=cmr11
\font\typo=cmr12
\font\top=cmr17
\font\addr=cmss12
\font\it=cmti12
\font\bf=cmbx12
\baselineskip 13 true pt
\parindent 0.15 true in
\parskip 1 pt
\font\bigbo=cmbx12 scaled\magstep1
\def\bpar {\noindent\hangafter=1\hangindent=0.5 true in}
\def\bbpar {\hangafter=1\hangindent=1.2 in}
\def\msrule{\medskip\hrule\hrule\smallskip}
\typo
%%%%%%%%%%%%%%%%%%%%%%%%%%%%%%%%%%%%%%%%%%%%%%%%%%%%%%%%%%%%%%%%%%%%%%%
%\begin{document}



\vskip -1.6in
\centerline{\bigbo A.\ \     PROJECT SUMMARY}
\medskip

%\leftline{\bf              A.1 Science Overview                     }


The proposed program will deliver ...

As an integral
part of this project, we will engage a large number of undergraduate students in 
publication-quality hands-on research involving state-of-the art data mining and machine 
learning techniques.


\vskip 4pt
\leftline{\bf   A.1\ \ Intellectual merit of the proposed activity     }

The deliverables of this proposal will have an impact on the community that is 
much broader than its immediate goals. We will ...

\vskip 3pt
\leftline{\bf   A.2\ \  Broader impacts of the proposed activity        }

The educational component of this project will contribute to the training of next generation 
of ...

\vfil\eject
\end
%\end{document}

II populations to a distance of 30 kpc, as well as their light-curve based
metallicity distribution measurements. We will perform follow up time-resolved photometric 
observations of a subset of these stars with a dual purpose: to provide more 
accurate characterization of selection efficiency and metalicity accuracy, and 
to engage a large number of undergraduate students in publication-quality 
hands-on research. With several hundred observations per source, for about 
20 million sources, the LINEAR photometric database is the best
such resource currently available, and we propose to publicly release it, 
toghether with appropriate querying tools. 



\section{{\bf  INTRODUCTION}}

This three-year project aims to deliver improved observational constraints 
and model-based understanding for the formation and evolution of the 
Milky Way's disk. The project will capitalize on {\bf i)} the unprecedented data 
set obtained by WISE (Wide-field Infrared Space Explorer) space mission, and 
{\bf ii)} results produced during the PI's NSF-supported projects ``Towards a 
Panoramic 7-D Map of the Milky Way'' (completed in 2011) and ``Mapping the 
Milky Way: Data-miners, Modelers, Observers, Unite!'' (in its final third year). 
The work proposed here has two broad goals:

{\bf 1.} We will utilize a major new database of infrared observations obtained by
WISE, as well as various extant training samples, to develop robust and efficient methods for 
selecting Asymptotic Giant Branch stars (AGB; discussed in more detail in section 
\ref{sec:AGBintro}) using WISE photometry. AGB stars
are sufficiently numerous in the Galaxy (of the order 10$^5$) to enable detailed
studies of its structure. They can be detected by WISE to distances exceeding 100 kpc,
with good photometric distance estimates.
{\it Most importantly}, WISE can detect them throughout the
disk with only a minor influence of the interstellar dust extinction (which is
more than ten times smaller in WISE bands than in the visual $V$ band). 
Using stellar population models developed to interpret SDSS (Sloan Digital Sky Survey)
and 2MASS (Two Micron All Sky Survey) data, 
and circumstellar dust emission models for AGB stars, we will quantify the selection
efficiency and completeness functions of the resulting selection methods. This project 
goal will serve a dual purpose: it will enable the use of WISE-selected samples 
of AGB stars for Galactic structure studies and 
it will engage a large number of undergraduate students in publication-quality 
hands-on research involving state-of-the art data mining and machine learning techniques. 
% (of the order $(2-3)\times10^5$ stars),

{\bf 2.} We will synthesize these new results based on WISE AGB stars, and recent 
SDSS- and 2MASS-based results on mapping of the spatial, kinematic and metallicity 
distributions of the Milky Way main sequence stars, into a self-consistent observational
picture. The two most important aspects of inferences based on AGB stars are
the essentially full coverage of the Galactic disk and the ability to bracket particular 
epochs in the disk history due to fairly narrow age distributions of appropriately chosen
AGB subsamples (compared to main sequence stars). We will combine these observational 
results with fast advances in cosmologically motivated N body and other models for disk 
formation and evolution to critically compare data and 
theory by asking questions such as: Do model galaxies look similar to the observed Milky Way? 
Can state-of-the-art  models produce stellar spatial, kinematic, metallicity and age 
distributions in agreement with recent SDSS, 2MASS, WISE and other data sets?
Can popular thick disk formation scenarios, such as radial migration and gas-rich mergers, 
explain anticipated differences in the distributions of main sequence and AGB stars? 
Does the behavior of AGB stars support a picture, motivated by SDSS data, where
chemical composition parameterized by measured metallicity ($[Fe/H]$) and 
$\alpha$-element abundance ($[\alpha/Fe]$) is a good age tracer? If so, can we use 
models to learn about quantities that are not readily observed, such as detailed
age distributions and Galactic gravitational potential? Can we use the same models to 
decipher complex correlations among various observables uncovered in modern 
Milky Way surveys? 



These deliverables will have an impact on the community that is much broader 
than the focus of this proposal. The expected two hundred
thousand candidate AGB stars present in WISE dataset will enhance our
understanding of the stellar mass-loss process and will form the basis for selecting
well-controlled subsamples for followup with other facilities (e.g. ALMA, or
photometric observations to characterize light curves, which is a popular 
endeavor among amateur astronomers). The resulting AGB sample will also 
enable characterization of the distribution of the ISM dust to distances 
significantly exceeding the current limits. The circumstellar dust opacity 
models will be improved at WISE wavelengths.  The improved understanding 
of the  stellar component in WISE database will help with the analysis of extragalactic 
populations. Last but not least, the educational component of this project will
contribute to the training of next generation of astronomers to handle massive 
data sets and utilize modern data mining methods. 





The research proposed in this work is heavily based on a new reality
in observational astronomy: the availability of science-ready large
complex astronomical databases. Such readily available resources are
having a strong impact on areas ranging from high school education to graduate
training of future astronomers. Data mining of these new datasets is particularly
powerful when combined with follow up observations and analysis of 
sophisticated numerical models. At the same time, there is strong evidence that 
research experiences for undergraduates have exceedingly strong impact on 
students' decision to stay in the sciences (Rothman \& Narum 1999). 
Taken together, these facts motivate our proposal to combine our
scientific goals with an effort to provide meaningful research experiences
for undergraduate students at the University of Washington. The essential
features of the proposed program have already been extensively used and
tested in practice as part of other PI's programs supported by NSF (listed
in section \ref{sec:priorNSF}). 

The main features of our program are: 

{\bf Expose students to data-mining in astronomy:} 
we will use SDSS SkyServer exercises and the WISE database to 
train students in statistical analysis of large datasets. For example, 
students will be required to download WISE data,  correlate them with 
SDSS and 2MASS databases, and construct and interpret appropriate
color-magnitude and color-color diagrams. Students will use an
extensive set of publicly available modern python tools developed in 
support of a textbook on data mining and machine learning in astronomy
coauthored by the PI (Princeton University Press, 2013). 

{\bf Educate students about state-of-the-art numerical models
of the Universe:}  we will place their research experience in a broader
context by discussing the predictions of Galfast and N-body galaxy 
models, and how new observations could lead to support for, or rejection
of, such models. 

{\bf Expand students' experience with professional astronomy:} all students 
will participate in the full publication process, starting with discussions of rough 
ideas, early data reduction and paper drafts, all the way to writing the last reply 
to the referee. 

Similar programs undertaken as parts of the PI's earlier projects have been
very successful; over the last five years, over twenty undergraduate students
have participated in the research and publication process (including authorship)
under PI's guidance, and about twice as many contributed to smaller
research units. 





1008784	Standard Grant	Mapping the Milky Way: Data-miners, Modelers, Observers, Unite!	10/01/2010	09/30/2013
                               
This award with this amendment totals $439,447 and expires September 30, 2013.  



                                
                                 


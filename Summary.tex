\documentclass[nofootbib,floatfix,11pt]{article}
\usepackage[square]{natbib}
\usepackage[paperwidth=8.5in,paperheight=11in,centering,margin=1in]{geometry}
\usepackage{paralist}
\usepackage{parskip}

\setlength{\parindent}{0pt}

\usepackage[compact,medium]{titlesec}
\titlespacing{\part}{0pt}{*0}{2ex}
\titlespacing{\section}{0pt}{2pt}{1ex}
\titlespacing{\subsection}{0pt}{2pt}{1ex}
\titlespacing{\subsubsection}{0pt}{*0}{1ex}
\titleformat*{\section}{\large \bf}
\titleformat*{\subsection}{\bf}
\titleformat*{\subsubsection}{\itshape}

% suck up extra white space
\setlength{\parskip}{3pt}
\setlength{\parsep}{0pt}
\setlength{\headsep}{0pt}
\setlength{\topskip}{0pt}
\setlength{\topmargin}{0pt}
\setlength{\topsep}{0pt}
\setlength{\partopsep}{0pt}

\renewcommand{\thefootnote}{\alph{footnote}}
\setlength{\belowcaptionskip}{-10pt}

\usepackage{amsmath}
\usepackage{amsbsy}

\usepackage{epsfig}
\usepackage{color}
\usepackage{subfigure}
\usepackage{graphicx}

%commented out the line below to get rid of citations that run into margins
%\usepackage{multicol}

%\usepackage{etoolbox}

%\renewcommand{\baselinestretch}{0.971}

\usepackage{hyperref}


\title{Statistical Explorations of Data in the {ACMS} {Program} at the
  {University} of {Washington}}
\author{PI: Marina \meila, Co-PIs: Andrew Connolly, \v{Z}eljko Ivezi\'{c}, Jacob Vanderplas}

%%%%%%%%%%%%%%%%%%%%%%%%%%%%%%%%%%%%%%%%%%%%%%%%%%%%%%%%%%%%%%%%%%%%%%%
\font\math = cmmi12
\newcommand\x         {\hbox{$\times$}}
\hyphenation{para-met-ri-zed Qua-sars cross-vali-da-tion}
%%%%%%%%%%%%%%%%%%%%%%%%%%%%%%%%%%%%%%%%%%%%%%%%%%%%%%%%%%%%%%%%%%%%%%%%%%%%%%%%
%% Commands added by MMP 
\newenvironment{itemize*}{
\begin{itemize}
\setlength{\parskip}{0em}
\setlength{\topskip}{0em}
}
{\end{itemize}}

\newenvironment{enumerate*}{
\begin{enumerate}
\setlength{\parskip}{0em}
\setlength{\topskip}{0em}
}
{\end{enumerate}}

\newcommand{\comment}[1]{}
\newcommand{\mmp}[1]{\textcolor{red}{#1}}

\newcommand{\meila}{Meil\u{a}}
\newcommand{\cdse}{CDS\&E}

\newcommand{\astroml}{{\em AstroML}}
\newcommand{\astrocl}{{\sc ASTR 497}}
\newcommand{\statcl}{{\sc STAT 391}}

\newcommand{\bit}{\begin{itemize}}
\newcommand{\eit}{\end{itemize}}
\newcommand{\bits}{\begin{itemize*}}
\newcommand{\eits}{\end{itemize*}}
\newcommand{\benum}{\begin{enumerate}}
\newcommand{\eenum}{\end{enumerate}}
\newcommand{\benums}{\begin{enumerate*}}
\newcommand{\eenums}{\end{enumerate*}}


%%%%%%%%%%%%%%%%%%%%%%%%%%%%%%%%%%%%%%%%%%%%%%%%%%%%%%%%%%%%%%%%%%%%%%%%%%%%



\begin{document}

%\vskip -1.6in
%\centerline{\Large     PROJECT SUMMARY}
%\medskip

Most disciplines, from the physical to the life sciences, have entered
an era, where discovery is no longer limited by the collection and
processing of data, but by the management and the statistical analysis
of this information. Novel developments in instrumentation have lead
to a tremendous increase in the magnitude of these data (leading to a
focus on \emph{Big Data}).  While significant steps in the development
of the statistical methodologies for processing Big Data have been
made, these ``hammers'' are rarely tied to fundamental questions that
science and society wish to address. There remains a disconnect
between methodological approaches to Big Data and the science that
they can enable.
%The goal of this proposal is to address this disconnect
%through the development of a educational program for big data in
%statistics and mathematics.

\noindent{\bf Education and Training:}
The goal of this EXTREEMS-QED program is to create a new breed of
statistician: proficient in and able to understand the tools and
methodologies for data-intensive statistics yet able to apply these
tools and models to some of the biggest scientific questions of the
coming decades.  Most importantly, this program will have an immutable
focus on multidisciplinary training, thus blurring the distinctions
between students of mathematics, statistics, and the domain sciences.

Our transformative path to address the challenges associated with
these goals comprises: the development of a Big Data track as part of
the ACMS (Applied and Computational Math Sciences) Program at the
University of Washington, tying the statistical methodologies to
real-world science challenges (initially Astrophysics and then a
broader range of domain sciences), and leveraging partnerships with
the eScience Institute, the UW IGERT graduate program in Big Data, and
the dual track PhD program in Computer Science and Statistics. 

The Big Data track will comprise two elective courses an Undergraduate
\cdse~ Research Seminar. The speakers in this seminar series will be
undergraduates themselves or their mentors, and the seminar topics
will be their research projects. With continuous evaluation of the
impact of these courses we expect to redesign the track to incorporate
two groups: math/stat electives (group I) and computing and science
electives (group II) with focuses on the theory or methodology and
application respectively.

%This educational program fits naturally with a number of initiatives
%to integrate the mathematical and physical sciences around the
%concepts of ``Big Data'', that are ongoing at the University of
%Washington. Prime amongst these endeavors are the creation of an
%eScience Institute, and a new NSF-sponsored IGERT graduate student
%program in ``Big Data''. We will leverage these programs throughout
%this initiative by enhancing the curriculum and research experiences
%of the mathematical and statistical students in data-intensive
%science, by providing the resources to enable hands-on research
%experiences, and by engaging the IGERT graduate students in working
%with the undergraduate students.

\noindent{\bf Research:}
Our primary research goal is to introduce and prepare students to
perform research in computationally intense statistical modeling and
in the statistical analysis of large scientific data. Based on the
organizational structure of the successful University of Washington's
Pre-Map Program (``Pre-Major in Astronomy'') we will develop small
teams including graduate and undergraduate students (with the
recruitment process targeted at minorities and underrepresented
groups).  The student teams will be mentored by the four PI's and will
work on real research projects including: non-linear dimension
reduction in large scientific data, and non-parametric clustering and
density estimation. To enable students to focus on the statistical
application as opposed to the software development, initial research
projects will be based on existing scalable Python applications,
developed by the coPIs, that are suitable for large data sets.

\noindent{\bf Faculty Professional
Development/Outreach:}
The development of the ACMS Big Data track will in of itself have a
broad impact on the University and on the teaching and impact of
statistics for data-intensive sciences. It will enable opportunities
for undergraduate research and hands-on experiences centered on \cdse\
and will result in significant changes to the undergraduate
mathematics and statistics curriculum at the University of Washington
(with broad institutional support and buy-in from two major
departments; Statistics and Astronomy). Additional outreach and
development activities include a workshop centered on professional
development activities for faculty from other institutions wishing to
emulate the proposed program. All course material, software, and
evaluations will be made available as part of this project. 

\end{document}

\documentclass[nofootbib,floatfix,11pt]{article}
\usepackage[square]{natbib}
\usepackage[paperwidth=8.5in,paperheight=11in,centering,margin=1in]{geometry}
\usepackage{paralist}
\usepackage{parskip}

\setlength{\parindent}{3ex}

\usepackage[compact,medium]{titlesec}
\titlespacing{\part}{0pt}{*0}{2ex}
\titlespacing{\section}{0pt}{2pt}{1ex}
\titlespacing{\subsection}{0pt}{2pt}{1ex}
\titlespacing{\subsubsection}{0pt}{*0}{1ex}
\titleformat*{\section}{\large \bf}
\titleformat*{\subsection}{\bf}
\titleformat*{\subsubsection}{\itshape}

% suck up extra white space
\setlength{\parskip}{3pt}
\setlength{\parsep}{0pt}
\setlength{\headsep}{0pt}
\setlength{\topskip}{0pt}
\setlength{\topmargin}{0pt}
\setlength{\topsep}{0pt}
\setlength{\partopsep}{0pt}

\renewcommand{\thefootnote}{\alph{footnote}}
\setlength{\belowcaptionskip}{-10pt}

\usepackage{amsmath}
\usepackage{amsbsy}

\usepackage{epsfig}
\usepackage{color}
\usepackage{subfigure}
\usepackage{graphicx}

%commented out the line below to get rid of citations that run into margins
%\usepackage{multicol}

%\usepackage{etoolbox}

%\renewcommand{\baselinestretch}{0.971}

\usepackage{hyperref}


\title{Statistical Explorations of Data in the {ACMS} {Program} at the {University} of {Washington}}
\author{PI: Marina \meila, Co-PIs: Andrew Connolly, \v{Z}eljko Ivezi\'{c}, Jacob Vanderplas}

%%%%%%%%%%%%%%%%%%%%%%%%%%%%%%%%%%%%%%%%%%%%%%%%%%%%%%%%%%%%%%%%%%%%%%%
\font\math = cmmi12
\newcommand\x         {\hbox{$\times$}}
\hyphenation{para-met-ri-zed Qua-sars cross-vali-da-tion}
%%%%%%%%%%%%%%%%%%%%%%%%%%%%%%%%%%%%%%%%%%%%%%%%%%%%%%%%%%%%%%%%%%%%%%%%%%%%%%%%
%% Commands added by MMP 
\newenvironment{itemize*}{
\begin{itemize}
\setlength{\parskip}{0em}
\setlength{\topskip}{0em}
}
{\end{itemize}}

\newenvironment{enumerate*}{
\begin{enumerate}
\setlength{\parskip}{0em}
\setlength{\topskip}{0em}
}
{\end{enumerate}}

\newcommand{\comment}[1]{}
\newcommand{\mmp}[1]{\textcolor{red}{#1}}

\newcommand{\meila}{Meil\u{a}}
\newcommand{\cdse}{CDS\&E}

\newcommand{\astroml}{{\tt AstroML}}
\newcommand{\astrocl}{{\sc Astr 497}}
\newcommand{\statcl}{{\sc Stat 391}}

\newcommand{\bit}{\begin{itemize}}
\newcommand{\eit}{\end{itemize}}
\newcommand{\bits}{\begin{itemize*}}
\newcommand{\eits}{\end{itemize*}}
\newcommand{\benum}{\begin{enumerate}}
\newcommand{\eenum}{\end{enumerate}}
\newcommand{\benums}{\begin{enumerate*}}
\newcommand{\eenums}{\end{enumerate*}}


%%%%%%%%%%%%%%%%%%%%%%%%%%%%%%%%%%%%%%%%%%%%%%%%%%%%%%%%%%%%%%%%%%%%%%%%%%%%



\begin{document}

\centerline{\large \bf Data Management Plan}

There will be two types of data that we will manage and make publicly avaialable.

First, we will make extensive use of catalogs delivered by modern astronomical
surveys (e.g. SDSS, but also many others). We will publish all SQL queries used to 
download data from the various databases together with other public project documentation.
All the processing of raw downloaded data will be performed with a suite of python
tools. Many of these tools (mostly data mining and machine learning applications,
but also various visualization algorithms) were already placed in the public domain 
(available from http://astroml.github.com/); those that will be developed for
this project will also be made freely available via github. 

In principle, all data products will be exactly reproducible by anyone, given that all
our SQL queries and post-processing software will be published. Nevertheless, it is 
often beneficial to others to also have access to the actual final data products. Hence,
all catalogs and other metadata resulting from this project will be made accessible via 
the Co-PI Ivezi\'{c}'s ``Data Depot'' site (available from the PI's home webpage under 
pull-down menu ``Research -- Data Depot''),  which already includes a large collection 
of data products from prior projects. The preferred data formats include both FITS binary 
tables and plain ascii text files (when practical). 

In addition, we expect to collect much smaller but valuable dataset based on evaluation 
services for this project provided by the University of Washington Office of Educational 
Assessment (OEA). We anticipate that these data will be used by others who will emulate
our program and we will make them available, too.

\end{document} 



\subsection{Leveraging Current NSF Funding}

Co-PI Vanderplas is currently supported by a 3-year NSF postdoctoral
fellowship through the interdisciplinary CI-TraCS program.  Though
Vanderplas' background is in Astronomy, the sponsoring professor is in
the Computer Science and Engineering department.  The focus of the
fellowship is research on the computational side of Astronomy, especially
on efficient statistical analysis of very large datasets.

A full 20\% of the fellowship time is devoted to teaching and course
preparation, and as part of this requirement Vanderplas has developed
and taught a Fall 2013
graduate seminar course through the Astronomy department:
Astr 599, {\it Scientific Computing with Python}.  The purpose of the
course is to offer a comprehensive introduction to scientific computing
in the Python programming language, geared toward graduate 
and advanced undergraduate students in Astronomy.
After stepping through the fundamental tools of scientific computing,
the course scratches the surface of statistical, machine learning, and
datamining methods made available through various packages in the scientific
Python ecosystem. The entirety of
the curriculum material is made available on the course website\footnote{
\url{http://www.astro.washington.edu/vanderplas/Astr599/}}.
In the remaining two years of his fellowship, Vanderplas will expand this
curriculum and offer the course to a wider audience of students through
the University's inter-disciplinary eScience Institute.

This curriculum is in many ways a fundamental component of the goals of the
current proposal.  Practical statistical analysis and data mining 
requires a certain level of proficiency in a scientific computing platform:
this course equips students with that foundational knowledge from which they
can explore the use of data mining and machine learning algorithms within
their own field.

As the current proposal moves forward, we will...

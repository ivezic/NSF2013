
\section{    {\bf        PROJECT ORGANIZATION         }}
\label{plan}

\subsection{                      Key Project Aims                   }
 
This project will  {\bf i)}  develop something, {\bf ii)} 
apply these methods, and {\bf iii)}  synthesize and compare...

These deliverables will have an impact on the community that is much broader 
than the focus of this proposal. 


\subsection{Responsibilities and Schedule}

The PI, Meila, will be responsible for the overall success of the project.


The co-PIs, Connolly, Ivezi\'{c}, and Vanderplas will be grossly irresponsible. 


{\bf In summary, the project schedule is:}

{\bf Year 1:}
Think

{\bf Year 2:}
Do

{\bf Year 3:}
Analyze


\subsection{             Results from Prior NSF Support             }
\label{sec:priorNSF}

The Co-PI Ivezi\'{c} was recently PI on four projects supported by NSF that
are indirectly related to the work proposed here (mostly through data mining
aspects, public release practice for all data products, and through engaging large 
numbers of students in research and publication process). 

The projects ``Towards a Panoramic 7-D Map of the Milky Way'' (AST-070790) and 
``Mapping the Milky Way: Data-miners, Modelers, Observers, Unite!'' (AST-1008784)
quantified statistical behavior of a few tens of millions of Milky Way stars observed by 
the Sloan Digital Sky Survey in multi-dimensional position--velocity--chemical composition 
space. The results were published in over a dozen refereed papers, and the work engaged
four graduate students (including two Ph.D. theses) and 11 undergraduate students.
A team of three undergraduate students has developed an education and public outreach site.

The key project aims for the NSF award AST-0507529 ``Interpretation of Modern Radio 
Surveys: Test of the Unification Paradigm'' were unification of several modern radio 
catalogs into a single public database containing several million sources and 
morphological classification of the matched sources. This three-year long project has 
produced six journal publications, two Ph.D. theses, and has engaged six undergraduate 
students in data analysis and publications. 

The project ``Statistical Description and Modeling of the Variability of Optical Continuum 
Emission from Quasars'' (AST-0807500) used time-domain data for the exploration of quasar 
physics. This three-year project has produced four journal publications, a Ph.D. thesis, and has 
engaged four undergraduates in data analysis and publications. 

\end{document} 



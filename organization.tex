
\section{    {\bf        PROJECT ORGANIZATION         }}
\label{plan}

\subsection{                      Key Project Aims                   }
 
This project will  {\bf i)}  develop something, {\bf ii)} 
apply these methods, and {\bf iii)}  synthesize and compare...

These deliverables will have an impact on the community that is much broader 
than the focus of this proposal. 


\subsection{Responsibilities and Schedule}

The PI, Meila, will be responsible for the overall success of the project.


The co-PIs, Connolly, Ivezi\'{c}, and Vanderplas will be grossly irresponsible. 


{\bf In summary, the project schedule is:}

{\bf Year 1:}
Think

{\bf Year 2:}
Do

{\bf Year 3:}
Analyze


\subsection{             Results from Prior NSF Support             }
\label{sec:priorNSF}

\meila, Connolly, Ivezi\'{c} , and van der Plas are all part of an active
and ongoing collaboration between statisticians, astrophysicists, and
computer scientists at the University of Washington and Carnegie
Mellon University.  This collaboration, dating to 2001, has
demonstrated sustained success and was cited by the President of the
American Statistical Association (ASA) as an exemplary
interdisciplinary research team \cite{straf03}. Highlights from this
collaboration include advances in cross-disciplinary education
(including the integration of research and teaching), establishing a
joint ``Big-Data'' PhD program between Computer Science and
Statistics, the creation of an NSF-sponsored IGERT program between
Statistics, Computer Science and the domain sciences on the topic of
``Big-Data'', and numerous advances in computational and statistical
approaches to data intensive astrophysics.

PI \meila's recent project ``Intransitive Game-Thoretic'' classifiers
(IIS-0535100) analyzed the preference and ranked data in the context
of intransitivity. This three year project produced\footnote{This
  counts only the products in which \meila~ was involved directly as
  co-author or advisor.}: 2 journal papers, 5 refereed conference
papers, 5 other publications, a code package available at
{www.stat.washington.edu/mmp/intransitive}, a jointly taught
Statistics/EE graduate course, a NIPS Workshop (co-organized by
\meila), 2 summer reading groups at UW and one at MIT. It has involved
5 statistics students, 1 CS student, 1 Applied Math student and 1
undergraduate. The grant ``Doctoral Student Forum and Student Travel
at the 2011 SIAM Data Mining Conference'' (DMS-1103263) supported the
travel of 14 out of town students to participate in the Doctoral Forum
(9 out of these being statistics students), and an additional 7
students in the conference. The Forum activities: panel discussion,
poster session, and best poster award were a succes. (In particular,
several students obtained internships, organized SDM workshops in
2013, and improved their papers as a result of the Forum). \meila~ is
a recipient of the grant ``Statistical Modeling the Functional
Activity in the Primary Motor Cortex'' from the {\em NSF ERC for
  Sensorymotor and Neural Engineering}, is a two year ongoing
project. It has involved so far 2 statistics graduate students and one
undergraduate.

Connolly is the Simulation Scientist for the LSST; responsible for the
modeling and simulations of the scientific performance of the
LSST. Connolly's most recent NSF award IIS-0844580 \$490,398 ``Putting
Astronomy's Head in the Cloud'' (2009 -- 2012) led to the development
of scalable image analysis tools that are built upon the Hadoop
platform\citep{wiley2011}. His work focuses on statistical approaches
to large astrophysical data sets including the development of n-tree
searching algorithms that make the calculation of n-point correlation
functions scale to size of current surveys \cite{Moore00}. This
software was made publicly available and has been used to compute the
2--point function on over 10$^6$ galaxies and the 3--point correlation
function of 400,000 galaxies from the SDSS survey
\cite{Scranton2002,Szapudi2002,Nichol2006,mcbride2011a,mcbride2011b}. This
work has resulted the introduction of signal compression and analysis
techniques to astronomy that are now regularly applied to the analysis
of spectroscopic surveys. His papers have demonstrated the ability of
computer algorithms to automatically classify astronomical objects
\cite{vdp2009,daniel2011} as well as using simplified active learning
methods to accelerate the exploration of complex parameter spaces
\cite{daniel2012}.

The Co-PI Ivezi\'{c} is project scientist for the LSST. He was
recently PI on four projects supported by NSF that are indirectly
related to the work proposed here (mostly through data mining aspects,
public release practice for all data products, and through engaging
large numbers of students in research and publication process).

The projects ``Towards a Panoramic 7-D Map of the Milky Way''
(AST-070790) and ``Mapping the Milky Way: Data-miners, Modelers,
Observers, Unite!'' (AST-1008784) quantified statistical behavior of a
few tens of millions of Milky Way stars observed by the Sloan Digital
Sky Survey in multi-dimensional position--velocity--chemical
composition space. The results were published in over a dozen refereed
papers, and the work engaged four graduate students (including two
Ph.D. theses) and 11 undergraduate students.  A team of three
undergraduate students has developed an education and public outreach
site.

The key project aims for the NSF award AST-0507529 ``Interpretation of
Modern Radio Surveys: Test of the Unification Paradigm'' were
unification of several modern radio catalogs into a single public
database containing several million sources and morphological
classification of the matched sources. This three-year long project
has produced six journal publications, two Ph.D. theses, and has
engaged six undergraduate students in data analysis and publications.

The project ``Statistical Description and Modeling of the Variability
of Optical Continuum Emission from Quasars'' (AST-0807500) used
time-domain data for the exploration of quasar physics. This
three-year project has produced four journal publications, a
Ph.D. thesis, and has engaged four undergraduates in data analysis and
publications.



Beyond the statistical and astrophysical aspects of the research
described above the PIs have demonstrated a commitment for integrating
research and education (see, for example, the extensive machine
learning tools available at http://astroml.github.com and the XXX). In
the field of informal education, Connolly was the technical lead for
the development of Sky in Google Earth (Google Sky;
http://earth.google.com) which enabled the seamless exploration of
astronomical images of the sky, and recently developed an affordable
digital planetarium system using Microsoft's World Wide Telescope
software \cite{rosenfield2011}.

All tools, codes and educational material developed through this
program will be open-source and made available on the web. 

\section{Cemetery}

To develop a software infrastructure for teaching \cdse~ in
  Python. This will include data sets, data analysis problems,
  software libraries, and course modules built around the data and
  problems. This infrastructure will be made available via the
  web. Due to its modular structure, it will be useable as needed by
  instructors in other courses.
 To organize an Undergraduate Research Seminar. In this seminar, unlike the ACMS 
 To organize a 3 day workshop for instructors in statistics and related fields that will teach the basics of using our sofware infrastructure and will impart our experience in the project.



\mmp{Zeljko's ``key words'' to scatter in the text moved here from the Introduction}
{\it 
1) course/class work 

- we will contribute to education of the next generation of mathematics and statistics 
   undergraduate students to confront new challenges in computational and data-enabled 
   science and engineering (\cdse) 

- we will also include math and stat minors 

- our efforts will result in significant changes to the undergraduate curriculum

- student training will incorporate computational tools for analysis of large data sets and
   for modeling and simulation of complex systems

- we will incorporating \cdse content in existing courses and develop new courses in 
   \cdse areas

- we will create resources for scientific education, including cyber-enabled pedagogies 
   (eBooks, online resources, etc.).

- we will foster interdisciplinary collaborations aiming to transform both departmental and 
   institutional culture.

- we have broad institutional support and department-wide commitment that encourage 
   collaborations within and across disciplines


2) research work

- research work will be broadly defined, long-term, team-based, interdisciplinary, and 
   will include with other institutions 

- we will development tools and theory for analyzing massive data sets

- we will use cyberinfrastructure to model and visualize complex scientific and engineering 
   concepts;

- we will create resources for scientific investigation, including state-of-the-art tools and 
  theory for knowledge discovery from massive, complex, and dynamic data sets

- we will foster interdisciplinary collaborations

- we will promote undergraduate research and hands-on experiences centered on \cdse

- the hands-on research work will developing CI competences (programming, data 
   management, simulation-building)

- we will leverage and advance the use of cyberinfrastructure resources (e.g. data archives, 
   networks, advanced computing systems, visualization environmnets) for data exploration

- we will address data-intensive scientific problems (arising in astronomy and ...)

3) workshop

- professional development activities centered on \cdse for faculty or K-12 teachers

- we will foster interdisciplinary collaborations

- we will create new learning environments and experiences that immerse students in \cdse 
   while energizing and sustaining the professional growth of faculty in \cdse
}





\end{document} 



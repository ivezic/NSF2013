\section{PROJECT ORGANIZATION AND MANAGEMENT}
\label{plan}

\subsection{                      Key Project Aims                   }
\label{sec:key-aims}
 
%This proposal aims to 
\bits
\item Transform the ACMS (Applied and Computational Math Sciences) Program Statistics track
by developing two new courses and a research seminar. 
\item Expose students to computationally intensive data analysis and mentor them in research. 
\item Facilitate adoption of this program by other institutions.
\eits  
{\bf Long-term outcomes} For all participating students: 
continued involvement with \cdse~ research, 
declaration of newly developed computationally minded statistics major,
intentions to pursue or progression along academic and/or career path within \cdse~ (e.g., graduate school application, internships, employment placement). 
For UW curriculum: establishment of computationally minded statistics major,
increased enrollment in newly developed major, inclusion of developed activities in other classes or curricula.


\subsection{Responsibilities and Schedule}

\noindent Our main goals and deliverables are: 
\bits
\item {\bf Education and Training} (Section \ref{sec:education}).

  To transform the ACMS Program Statistics track to address the
  challenges of data intensive statistics including:
\begin{itemize}
  \item two new courses
  \item a  new data intensive research and education curriculum 
  \item an undergraduate \cdse~ Research Seminar
  \end{itemize}
\item {\bf Research and mentoring for research} (Section \ref{sec:research})
To facilitate the entrance of Statistics students into computationally
intensive data analysis research by:
\begin{itemize}
  \item the mentoring of undergraduate students by PI, Co-PIs, and graduate students
  \item the mentoring of Statistics and Astronomy graduate students by PI and Co-PIs
  \end{itemize}
\item {\bf Dissemination and Outreach} (Section \ref{sec:outreach})
 The creation of software and exemplar datasets for teaching
 computational statistics and machine learning including developing
 material for:
 \bits
  \item a workshop for UW faculty % in the use of the infrastructure and our experience with it
  \item workshop with outside participation. 
 \eits
\eits  

The PI, Meila, will be responsible for the overall success of the project. She will also be 
responsible for the course STAT 391, and for mentoring Statistics students. The Co-PIs 
Connolly and Ivezi\'{c} will be responsible for the course ASTR 497, and for mentoring 
Astronomy students. The Co-PI VanderPlas will be responsible for Python materials, 
and for maintaining and developing tools available through the {\it
  astroML} package.  The work will adhere to the following schedule: 
\bits
\item {\bf  Year 1:} Design and introduce two new courses, \statcl, \astrocl, to
the ACMS Program Statistics track. Start the Undergraduate \cdse\ Research Seminar. 
\item {\bf Year 2:} Complete the redesign of the ACMS Program Statistics track
by reorganizing the core and electives. Organize a workshop for UW faculty. 
Continue developing the courses' software and data infrastructure. 
\item {\bf Year 3:} Incorporate feedback from evaluators and all participants. 
Continue developing the courses' software and data infrastructure. 
Organize a workshop for participants from other institutions aiming to 
emulate the UW program.
\eits

\subsection{             Results from Prior NSF Support             }
\label{sec:priorNSF}

\meila, Connolly, Ivezi\'{c} , and Vanderplas are all part of an active
and ongoing collaboration between statisticians, astrophysicists, and
computer scientists at the University of Washington and Carnegie
Mellon University.  This collaboration, dating to 2001, has
demonstrated sustained success and was cited by the President of the
American Statistical Association (ASA) as an exemplary
interdisciplinary research team \cite{straf03}. Highlights
%from this collaboration
include advances in cross-disciplinary education
(including the integration of research and teaching), establishing a
joint ``Big-Data'' PhD program between Computer Science and
Statistics, the creation of an NSF-sponsored IGERT program between
Statistics, Computer Science and the domain sciences on the topic of
``Big-Data'', and numerous advances in computational and statistical
approaches to data intensive astrophysics.

PI \meila's recent project ``Intransitive Game-Thoretic'' classifiers
(IIS-0535100) analyzed the preference and ranked data in the context
of intransitivity. This three year project produced\footnote{This
  counts only the products in which \meila~ was involved directly as
  co-author or advisor.}: 2 journal papers, 5 refereed conference
papers, 5 other publications, a code package available at
{www.stat.washington.edu/mmp/intransitive}, a jointly taught
Statistics/EE graduate course, a NIPS Workshop (co-organized by
\meila), 2 summer reading groups at UW and one at MIT. It has involved
5 statistics students, 1 CS student, 1 Applied Math student and 1
undergraduate. The grant ``Doctoral Student Forum and Student Travel
at the 2011 SIAM Data Mining Conference'' (DMS-1103263) supported the
travel of 14 out-of-town students to participate in the Doctoral Forum
(9 out of these being statistics students), and an additional 7
students in the conference. 
%The Forum activities: panel discussion,
%poster session, and best poster award were a succes. (In particular,
%several students obtained internships, organized SDM workshops in
%2013, and improved their papers as a result of the Forum). 
\meila~ is a recipient of the grant ``Statistical Modeling the
Functional Activity in the Primary Motor Cortex'' from the {\em NSF
  ERC for Sensorymotor and Neural Engineering}. It has involved so far 2 statistics graduate students and one
undergraduate.

Co-PI Connolly is the Simulation Scientist for the LSST; responsible
for the modeling and simulations of the scientific performance of the
LSST. Connolly's most recent NSF award IIS-0844580 \$490,398 ``Putting
Astronomy's Head in the Cloud'' (2009 -- 2012) led to the development
of scalable image analysis tools that are built upon the Hadoop
platform\citep{wiley2011}. His work focuses on statistical approaches
to large astrophysical data sets including the automated
classification of spectra \cite{vdp2009,daniel2011} and active
learning methods to accelerate the exploration of complex parameter
spaces \cite{daniel2012}.

Co-PI Ivezi\'{c} is the Project Scientist for the LSST. He was
recently PI on two projects supported by NSF that are indirectly
related to the work proposed here (mostly through data mining aspects,
public release practice for all data products, and through engaging
large numbers of students in research and publication process).  The
project ``Mapping the Milky Way: Data-miners, Modelers, Observers,
Unite!'' (AST-1008784) quantified statistical behavior of a few tens
of millions of Milky Way stars observed by the Sloan Digital Sky
Survey in multi-dimensional position--velocity--chemical composition
space. The results were published in over a dozen refereed papers, and
the work engaged four graduate students (including two Ph.D. theses)
and 11 undergraduate students.  A team of three undergraduate students
has developed an education and public outreach site.
%The key project aims for the NSF award AST-0507529 ``Interpretation of
%Modern Radio Surveys: Test of the Unification Paradigm'' were
%unification of several modern radio catalogs into a single public
%database containing several million sources and morphological
%classification of the matched sources. This three-year long project
%has produced six journal publications, two Ph.D. theses, and has
%engaged six undergraduate students in data analysis and publications.
The project ``Statistical Description and Modeling of the Variability
of Optical Continuum Emission from Quasars'' (AST-0807500) used
time-domain data for the exploration of quasar physics. This
three-year project has produced four journal publications, a
Ph.D. thesis, and has engaged four undergraduates in data analysis and
publications.

Co-PI Vanderplas is currently an NSF post-doctoral fellow, through the NSF
Transformative Computational Science using Cyberinfrastructure (CI TRaCS)
program (``Weak Gravitational Lensing in the Petabyte Era'', award
CI-TraCS-1226371).
Vanderplas
%' PhD is in Astronomy, the fellowship is in the Computer
%Science and Engineering department's Database Research Group, and 
is focused on pushing the boundaries of data-intensive cosmological science through
fundamental work on new database technologies. The cross-disciplinary nature
of this work extends that of Vanderplas' past research, which includes several
projects with collaborators from computer science and related fields
\cite{scikit-learn2, Xiong2011, daniel2011, scikit-learn1}.
Vanderplas has spent a significant amount of time on the development of
open source scientific computing software, primarily in the Python
language.  Among other contributions,
he is a maintainer of the well-known compendium of scientific tools in
{\it SciPy}\footnote{\url{http://scipy.org}}, a core contributer to the
popular machine learning toolkit
{\it Scikit-learn}\footnote{\url{http://scikit-learn.org}},
and the primary author of several specialized Python packages,
including the {\it AstroML} package which is an integral piece of the
curriculum for the current proposal.
%As a result of his role in the scientific Python community, Vanderplas
%has presented dozens of talks, tutorials, and seminars related to scientific
%computing in Python, primarily at the
%PyCon\footnote{\url{https://us.pycon.org/}},
%SciPy\footnote{\url{http://conference.scipy.org/}},
%and PyData\footnote{\url{http://pydata.org/}} conference series.
This expertise contributes to the practical side of the above-mentioned
Astronomy textbook Vanderplas authored with Co-PIs Ivezic and Connolly.

The PIs have demonstrated a commitment for integrating research and
education (see, for example, the extensive machine learning tools
available at \url{http://astroml.org}). In the field of informal
education, Connolly was the technical lead for the development of Sky
in Google Earth (Google Sky; http://earth.google.com) which enabled
the seamless exploration of astronomical images of the sky, and
recently with Vanderplas worked to develop and deploy an affordable
digital planetarium system using Microsoft's World Wide Telescope
software \cite{rosenfield2011}.

\comment{
\section{Cemetery}

To develop a software infrastructure for teaching \cdse~ in
  Python. This will include data sets, data analysis problems,
  software libraries, and course modules built around the data and
  problems. This infrastructure will be made available via the
  web. Due to its modular structure, it will be useable as needed by
  instructors in other courses.
 To organize an Undergraduate Research Seminar. In this seminar, unlike the ACMS 
 To organize a 3 day workshop for instructors in statistics and related fields that will teach the basics of using our sofware infrastructure and will impart our experience in the project.



\mmp{Zeljko's ``key words'' to scatter in the text moved here from the Introduction}
{\it 
1) course/class work 

- we will contribute to education of the next generation of mathematics and statistics 
   undergraduate students to confront new challenges in computational and data-enabled 
   science and engineering (\cdse) 

- we will also include math and stat minors 

- our efforts will result in significant changes to the undergraduate curriculum

- student training will incorporate computational tools for analysis of large data sets and
   for modeling and simulation of complex systems

- we will incorporating \cdse content in existing courses and develop new courses in 
   \cdse areas

- we will create resources for scientific education, including cyber-enabled pedagogies 
   (eBooks, online resources, etc.).

- we will foster interdisciplinary collaborations aiming to transform both departmental and 
   institutional culture.

- we have broad institutional support and department-wide commitment that encourage 
   collaborations within and across disciplines


2) research work

- research work will be broadly defined, long-term, team-based, interdisciplinary, and 
   will include with other institutions 

- we will development tools and theory for analyzing massive data sets

- we will use cyberinfrastructure to model and visualize complex scientific and engineering 
   concepts;

- we will create resources for scientific investigation, including state-of-the-art tools and 
  theory for knowledge discovery from massive, complex, and dynamic data sets

- we will foster interdisciplinary collaborations

- we will promote undergraduate research and hands-on experiences centered on \cdse

- the hands-on research work will developing CI competences (programming, data 
   management, simulation-building)

- we will leverage and advance the use of cyberinfrastructure resources (e.g. data archives, 
   networks, advanced computing systems, visualization environmnets) for data exploration

- we will address data-intensive scientific problems (arising in astronomy and ...)

3) workshop

- professional development activities centered on \cdse for faculty or K-12 teachers

- we will foster interdisciplinary collaborations

- we will create new learning environments and experiences that immerse students in \cdse 
   while energizing and sustaining the professional growth of faculty in \cdse
}
}

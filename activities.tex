\section{Activities}
\label{sec:activities}

\subsection{Grad students involvement}

For the statistics graduate students funded by this grant, I envision 
\bits
\item to fund several students for a relatively short time (2 quarters
  to 1 year) We adopt this ``rotation'' plan recognizing that developing tools .
(This ``rotation'' model will also assure that the ``API'' of our data infrastructure is truly functional, as each departing grad student will have to ensure the smoothe transition to her/his successor.)
\item Student helps develop the software and data infrastructure for
  the program. Searches for available data sets, curates them, writes
  preprocessing software if necessary, designes tasks and exercises.
\item In the same time, student gets practice and training with
  analyzing large data. 
\item The student can be advised by the PI, by another Statistics faculty, or by another UW scientist with interests in statistical analysis of big data. Gradually, a research problem is formulated, and the student focuses on the data and methodology relevant to this problem. After this ``apprenticeship'', some students continue their research supervised by other advisors. 
\item The students will also be strongly recommended to TA the STAT 391 class, thus rounding their preparation and self-confidence. 
\item This will enable stat PhD students to play useful roles in the other NSF funded initiatives at UW. For instance, in the CSNE, with with the PI is involved, where the data collected, far from reaching Tera byte sizes, represents a daunting challenge for the average Statistics graduate student who hasn't accquired a CS degree before. This plan also harmonizes with the IGERT plan of offering graduate students the experience of working with domain scientists, and with outside big data companies via internships. 
\item Finally, the graduate student will participate in the Undergraduate Research Seminar, and will mentor 1-2 undergraduates. 
 \eits

\subsection{PI Involvement}

\bits
\item First year: program coordination and curriculum planning. Within the department and between the other departments participating in ACMS. Preparation for phase 2 of the project happens now. 

Develops (with the RA and the co-PI's) a plan for the core Python numerical and data structures libraries to be taught/presented. 

Starts developing course notes for \statcl.

\item Second year: Evaluates the success of the first phase and
  incorporates lessons learned. Writes the bulk of the STAT 391 course
  notes. Supervises the undergraduate research seminar. Major work
  (with RA) selecting/curating the data sets to be used in
  \statcl. \mmp{will lighten this because the salary was shrunk}

\item Third year: Evaluation of the second phase and fine-tuning of
  the curriculum and program requirements. Explores possibilities to
  open this pathway to math majors.
 \eits

\subsection{Undergraduate involvement}

The PI and co-PI's have extensive track records of involving
undergrads in research. 

Undergraduates who take either of the courses will be involved in
research projects either (1) along with the funded graduate students,
superivsed by the PI, co-PI's and the graduate students or (2) in the
UW units that provide data sets.

All undergraduates involved in research under this project, along with
the graduate students, will participate in an {\em Undergraduate
\cdse  Research Seminar} \comment{(UNDRESS)} where they will present and
discuss their work.  

The seminar will be open to any other undergraduate students involved
in research with Statistics faculty who would like to participate, as
well as any other undergraduates across campus interested in \cdse~
research. The seminar's goal is to be a forum where research
experiences are shared, research teams are formed, new research
projects are started, and familiarity with the strategic and social
aspects of research is gained. 

In addition to technical presentations/discussion by undergraduates,
the seminar will include:
\bits
\item meetings where faculty (from sciences for example) present quantitative data analysis research projects to recruit interested undergraduates
\item  mentorship sessions by guest speakers, UW graduate students and faculty, on carreer options in research, applying to graduate school, how to work in a team, how to approach a potential research advisor, how to approach a new research project
\item presentations from outside speakers (e.g local buinesses Amazon,
  Microsoft,...) on carrerr opportunties in \cdse. We will assure to bring in women and underrepresented minorities role models as often as possible among the internal and external guest speakers.
\eits
Some of the seminar's activities and goals overlap with the {\em Statistics and Probability Association}'s\footnote{{\tt http://students.washington.edu/spassc/index.html}}, and with the PreMAP program's, and we plan to do these jointly, reinforcing the  benefits for all groups involved.



\section{PROGRAM ACTIVITIES}
\label{sec:activities}

\subsection{PI and Co-PIs Involvement}

\bits
\item First year: program coordination and curriculum planning (within the department and between the other departments participating in ACMS). Preparation for stage 2 of the project. 
Development of a plan for the core Python numerical and data structures libraries to be taught.
Beginning of the development of course notes for \statcl.

\item Second year: Evaluation of the success of the first phase and
  incorporation of the lessons learned. Development of the bulk of the STAT 391 course
  notes. Supervision of  the undergraduate research seminar. 
 Selecting/curating  (with RAs) of the data sets to be used in
  \statcl. Organization and execution of the workshop. 
%\mmp{will lighten this because the salary was shrunk}

\item Third year: Evaluation of the second phase and fine-tuning of
  the curriculum and program requirements. Exploration of the possibilities to
  open this pathway to math majors. Organization and execution of the workshop. 
 \eits

\subsection{Undergraduate involvement}
\label{sec:ugrad-seminar}
The PI and all co-PI's have extensive track records of involving
undergrads in research. Undergraduates who take either of the courses will be involved in
research projects either (1) along with the funded graduate students,
supervised by the PI, co-PI's and the graduate students or (2) in the
UW units that provide data sets.

All undergraduates involved in research under this project, along with
the graduate students, will participate in an {\em Undergraduate
\cdse\  Research Seminar} where they will present and
discuss their work.  
The seminar will be open to any other undergraduate students involved
in research with Statistics faculty who would like to participate, as
well as any other undergraduates across campus interested in \cdse~
research. The seminar's goal is to be a forum where research
experiences are shared, research teams are formed, new research
projects are started, and familiarity with the strategic and social
aspects of research is gained. 

In addition to technical presentations and discussions by undergraduate
students, the seminar will include:
\bits
\item meetings where faculty (from sciences for example) present quantitative data analysis research projects to recruit interested undergraduates
\item  mentorship sessions by guest speakers, UW graduate students and faculty, on career options in research, applying to graduate school, how to work in a team, how to approach a potential research advisor, how to approach a new research project
\item presentations from outside speakers (e.g local businesses such as Amazon,
  Microsoft, and Google) on career opportunties in \cdse. We will assure to bring in women and underrepresented minorities role models as often as possible among the internal and external guest speakers.
\eits
Some of the seminar's activities and goals overlap with the goals of {\em Statistics and Probability Association}\footnote{{\tt http://students.washington.edu/spassc/index.html}}, and the Pre-MAP program\footnote{\tt http://www.astro.washington.edu/users/premap/}, 
and we plan to do these jointly, reinforcing the  benefits for all groups involved.


\subsection{Graduate students involvement}
\label{sec:activities-grad}

For the statistics graduate students funded by this grant, we envision 
\bits
\item to fund several students for a relatively short time (2 quarters
  to 1 year). We adopt this ``rotation'' plan recognizing that developing 
tools cannot be the only part of a PhD work in statistics. 
This ``rotation'' model will also assure that the ``API'' of our data
infrastructure is truly functional, as each departing graduate student 
will have to ensure smooth transition to her/his successor.)
\item Students will help develop the software and data infrastructure for
  the program, search for available data sets, curate them, write
  preprocessing software if necessary, and design tasks and exercises. 
  At the same time, students will get training for analyzing large data sets. 
\item Students will be advised by the PI, by another Statistics faculty, or by another UW scientist with interests in statistical analysis of big data. Gradually, a research problem will be formulated and the student will
focus on the data and methodology relevant to the chosen problem. After this ``apprenticeship'', 
some students will continue their research supervised by other advisors. 
\item The students will also be strongly recommended to TA the STAT 391 class, thus rounding their preparation and providing self-confidence. 
\item This program will enable statistics PhD students to play useful roles in other NSF funded initiatives at UW. For 
%instance, in the CSNE, with with the PI is involved, where the data collected, far from reaching Tera byte sizes, represents a %daunting challenge for the average Statistics graduate student who hasn't accquired a CS degree before. T
example, this plan harmonizes with the IGERT plan of offering graduate students the experience of working with domain scientists, and with outside big data companies via internships. 
\item Finally, the graduate students will participate in the Undergraduate Research Seminar, and will mentor 1-2 undergraduates. 
 \eits



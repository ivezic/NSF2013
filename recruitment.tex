 \section{Recruitment, Mentoring, and Retention}

 The University of Washington has long played a leadership role in
 recruiting, and mentoring students including those from traditionally
 under-represented groups (including women, people with disabilities
 and under-represented minorities). To ensure a diverse range of
 applicants to the ACMS program we will actively recruit students from
 under represented groups (targeting incoming undergraduate students
 who have expressed an interest in statistics or math).  We will
 leverage the IGERT and Pre-Map programs described below; with the
 graduate students in the IGERT program acting as mentors and role
 models for the undergraduate students. Working with participating
 departments will enable us to leverage their recruitment efforts and
 best practices in order to attract strong applicants across the math
 and statistics domains (throughout this section we will describe
 novel and ongoing approaches to recruitment and retention).


\subsection{Recruiting Under-represented Groups}
 
One focus of this program is to increase the participation of
\textbf{traditionally under-represented groups} in the statistical and
math communities. To accomplish this we will engage undergraduate
students early in their career (including incoming and prospective
students); introducing them to research in the first year at UW and
providing the additional support and mentoring in the core skill sets
that they will need.  This role involves implementing pipeline
partnership programs with undergraduate departments, programs, and
faculty at local, state, and regional community and state colleges for
the purpose of identifying potential students (e.g.\ $\sim$30\% of UW
students come from two-year community colleges). Additionally, to
establish in-person recruitment connections, on a regular basis, the
IGERT graduate students will present at these institutes as a
recruitment opportunity.


Examples of ongoing initiatives upon which we will build include an
NSF Innovation through Institutional Integration (I3) award (Promoting
Equity in Engineering Relationships; PEERS). PEERS Leaders give
presentations to student groups about the impact of bias. Through the
seminar series, each year, we will request a visit from PEERS Leaders
so that all prospective faculty and mentors can learn how bias can
affect student success.  The University of Washington is very active
in the area of promoting and increasing the participation of {\bf
  students with disabilities}. For example, in the computing fields we
will develop a partnership with the AccessComputing Alliance
%(\url{http://www.washington.edu/accesscomputing/}) 
and the University
of Washington DO-IT program
%(\url{http://www.washington.edu/doit/}). 
The AccessComputing Team will provide mentoring and other activities
that support students with disabilities in their educational goals.

To recruit and retain {\bf women} into this program, we will leverage
several existing organizations and efforts.  Our key partner will be
the University of Washington Advance Program~\cite{}.  ADVANCE is
working on creating a diverse, thriving campus in which all faculty in
science, engineering and mathematics (SEM) receive the proper support,
flexibility and recognition to achieve her or his maximum
potential. Since the inception of UW ADVANCE Center for Institutional
Change in 2001, SEM departments have seen an overall 59\% increase in
the number of tenured or tenure-track women faculty. The national
percentage of women tenured or tenure-track engineering faculty is
currently 13.2\% compared to 21.3\% at UW. As part of their efforts,
ADVANCE provides educational materials on gender bias, which we will
share among all participants of the EXTREEMS-QED program (faculty and
students)

\subsection{Mentoring, Networking, and Retaining Students}

Mentoring and community building will begin immediately upon
acceptance to the program. We will build upon the success of the
``Pre-Major in Astronomy Program''\cite{garner2010diversity} that was
developed in the Astronomy department to engage under-represented
students in research early in their careers and to provide a platform
for supplementing the educational needs of these students with
addition tutoring and course work. The methodology of the program
centers on mentoring and peer support with leadership for the program
coming from the graduate students.  As an example of the impact of
such an approach, in 2008, the three founding members of the astronomy
PREMAP were awarded a Certificate of Excellence for their astronomy
diversity efforts by the National Society of Black Physicists. All
three students obtained their PhDs and moved on to faculty positions
or prize postdoctoral fellowships.

One important way to keep track of the progress of students and to
build community is to have seminars where current students give
presentations on their research on a rotating schedule. Students have
the opportunity to learn important professional communication skills,
in a nurturing environment, including explaining their science to
individuals outside of their domain.  They will be able to receive
feedback on their ideas. In addition, the faculty, through periodic
faculty wide meetings, will monitor each student’s progress.  During
these reviews, the mentoring relationships of the student will be
examined so that the faculty and peer mentor roles are active and
fulfilling.

We anticipate that throughout the ACMS course the students will be
exposed to a large interdisciplinary pool of faculty participants with
diverse ethnicity and gender backgrounds (e.g., the faculty
participants from the departments in this proposal are XX\% women).
Such broad exposure will enable students to think outside their
science domain's traditional set of solutions to intellectual
roadblocks.  Avoiding such obstacles is important to keeping students
engaged in their research and on the path to successful completion of
their degree.  The diversity of the faculty will also offer direct
contact with role models for success in the sciences.

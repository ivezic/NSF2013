\section{ INTRODUCTION}

Most disciplines, from the physical to the life sciences, have entered
an era where discovery is no longer limited by the collection and
processing of data, but by the management and the statistical analysis
of this information. Developments in instrumentation have led to a
tremendous increase in the magnitude of available data.
While significant steps in the development
of the statistical methodologies for processing this {\it Big Data} have been
made, these ``hammers'' are rarely tied to fundamental questions that
science and society wish to address. There remains a disconnect
between methodological approaches to Big Data and the science that
they can enable.

This three-year project is designed to 
%use the 
%EXTREEMS-QED
%(Expeditions in Training, Research, and Education for Mathematics and
%Statistics Through Quantitative Explorations of Data) program to
address these challenges by training a new generation of statistics
and mathematics undergraduate students to confront the challenges in
computational and data-enabled science and engineering (\cdse).  The
proposed work addresses and includes all the main required project
components required for EXTREEMS-QED: Education and Training,
Research, and Faculty Professional Development. Our program will
provide opportunities for undergraduate research and hands-on
experiences centered on \cdse\ and will result in significant changes
to the undergraduate mathematics and statistics curriculum at the
University of Washington. We have secured broad institutional support
and buy-in from two major departments (Statistics and Astronomy) to
initiate this program with the expectation that, over the three years
of this project, it will grow to encompass Mathematics and a broader
range of science domains. The dissemination of new curricula, software,
research and teaching materials, together with a faculty-oriented
workshop centered on professional development activities, will foster
growth both in the University itself and within external partners.





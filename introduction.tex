\section{{\bf  INTRODUCTION}}

This three-year project is an enthusiastic response to the EXTREEMS-QED (Expeditions in Training, Research, 
and Education for Mathematics and Statistics Through Quantitative Explorations of Data) solicitation for proposals
that aim to train the next generation of statistics and mathematics undergraduate students for confronting new
challenges in computational and data-enabled science and engineering (\cdse).  The proposed work addresses and 
includes all the main required project components: Education and Training, Research, and Faculty Professional 
Development. The proposed program will provide opportunities for undergraduate research and hands-on experiences 
centered on \cdse and will result in significant changes to the undergraduate mathematics and statistics curriculum 
at the University of Washington. We have secured broad institutional support and buy-in from two major departments 
(Statistics and Astronomy), and the proposed work includes a workshop centered on professional development activities
for faculty from other institutions wishing to emulate the proposed program. 

The Education and Training component will be centered around the Applied and Computational Math Sciences (ACMS) 
Program at the University of Washington. We will significanly enhance this program in the context of computational 
and data-enabled science and engineering by including two new specialized courses in the ACMS statistics track. 
The first course, STAT 391 ``Computational Statistical Modeling and Machine Learning'', will include essential statistical 
methodology, such as regression and probability models for discrete and continous data, as well as topics crucial for 
data-enabled science, such as classification and clustering. The second course, ASTR 497 ``Data Intensive Astronomy 
and Astrophysics'', will apply methods introduced in STAT 391 to contemporary massive datasets collected by modern 
astronomical sky surveys, and further expend them with domain-specific methodologies. These courses are designed 
(1) to give students a hands-on experience with statistical modeling through programming and performing real data 
analyses on a computer, and (2) to introduce students to the machine learning methodology in particular, with specific 
attention to the issues of big data, with astronomy as an attractive core science example. 

The Research component will build upon existing close collaboration between Statistics and Astronomy departments at the 
University of Washington, led by the PIs of this proposal. In addition to three faculty and an NSF postdoctoral Fellow, 
the proposed program will also include two graduate students and a large number of undergraduate students. We will 
utilize a suite of statistical and machine learning methods to attack a number of unsolved challenges in data-intensive 
astronomy posed by recent data avalanches coming from modern astronomical sky surveys. 

The proposed program, including course work and supporting research efforts, will represent a paradigm-shifting model  
that may be easily adapted by other institutions. To facilitiate such adoption, the Faculty Professional Development component 
will utilize several communication and dissemination techniques, culminating with a workshop for faculty from other 
institutions wishing to emulate the proposed program. The workshop, and a supporting website, will disseminate all the 
teaching materials (including datasets and code to perform hands-on research exercises) and the results of program 
effectiveness evaluation. 

In the remainder of this proposal, we describe the new courses in detail, how they fit within the ACMS program and the  
overall ``Big Data'' efforts within the University of Washington, and the budget and execution schedule for the proposed 
program.  



{\bf This is from an old email summarizing this solicitation; some
statements that we should disperse through the text at some point:} 

{\it 
1) course/class work 

- we will contribute to education of the next generation of mathematics and statistics 
   undergraduate students to confront new challenges in computational and data-enabled 
   science and engineering (\cdse) 

- we will also include math and stat minors 

- our efforts will result in significant changes to the undergraduate curriculum

- student training will incorporate computational tools for analysis of large data sets and
   for modeling and simulation of complex systems

- we will incorporating \cdse content in existing courses and develop new courses in 
   \cdse areas

- we will create resources for scientific education, including cyber-enabled pedagogies 
   (eBooks, online resources, etc.).

- we will foster interdisciplinary collaborations aiming to transform both departmental and 
   institutional culture.

- we have broad institutional support and department-wide commitment that encourage 
   collaborations within and across disciplines


2) research work

- research work will be broadly defined, long-term, team-based, interdisciplinary, and 
   will include with other institutions 

- we will development tools and theory for analyzing massive data sets

- we will use cyberinfrastructure to model and visualize complex scientific and engineering 
   concepts;

- we will create resources for scientific investigation, including state-of-the-art tools and 
  theory for knowledge discovery from massive, complex, and dynamic data sets

- we will foster interdisciplinary collaborations

- we will promote undergraduate research and hands-on experiences centered on \cdse

- the hands-on research work will developing CI competences (programming, data 
   management, simulation-building)

- we will leverage and advance the use of cyberinfrastructure resources (e.g. data archives, 
   networks, advanced computing systems, visualization environmnets) for data exploration

- we will address data-intensive scientific problems (arising in astronomy and ...)

3) workshop

- professional development activities centered on \cdse for faculty or K-12 teachers

- we will foster interdisciplinary collaborations

- we will create new learning environments and experiences that immerse students in \cdse 
   while energizing and sustaining the professional growth of faculty in \cdse
}




\section{Dissemination and Outreach}
\label{sec:outreach}


It is our strong desire to make our educational and research
program portable and adaptable by other institutions. For example, 
the lack of such a program that can be easily emulated was identified 
as one of the main impediments for wide adoption of astrostatistics
in graduate astronomy curricula around the country (by the Working 
Group on Astroinformatics and Astrostatistics of the American Astronomical
Society, chaired by the Co-PI Ivezi\'{c}).  We intend to make all our
materials public, including lectures for both courses, data sets used
in lectures and research seminars, and all the data analysis code. 


\subsection{Workshops} 
\label{sec:workshop}

In order to facilitate easy adoption of our program by others, we 
plan to hold two workshops. The first workshop, which will also 
serve as a dry run for the second larger workshop, will be limited to
faculty from the University of Washington, and it will be held during
the second program year. We anticipate about 10 faculty from various 
science and engineering departments attending this workshop. Given 
the local attendees, we plan to organize two half-day sessions focused 
on i) program outline and materials, and ii) measured program progress 
and ideas for improvements. 

The second workshop will be longer and open to everyone. We plan a 
three-day workshop and anticipate about 30 faculty from other
institutions of higher learning in attendance. The workshop program
will include i) the program presentation and discussion of available 
materials (day 1), and ii) hands-on work with available data sets and
tools (such as {\it astroML} exercises; day 2), and iii) a discussion
of measured program progress and ideas for improvements (half day, day
3). We have budgeted modest travel support for 4 young faculty and
postdocs, and lunch/coffee/conference dinner for 30 attendees. 
All the workshop materials and presentations will be made publicly
available via a website. 

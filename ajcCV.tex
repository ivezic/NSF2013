\documentclass[12pt,tightenlines]{revtex4} % 11pt font = 12pt Word font
\usepackage{color}
\usepackage{graphicx}
\usepackage{epsfig,wrapfig}
\usepackage{fancyhdr,lastpage}
\usepackage[T1]{fontenc}
\usepackage[latin1]{inputenc}
\usepackage[english]{babel}
\usepackage{helvet}

%%%%%%%%%%% Change This section as needed %%%%%%%%%%%
%%%%%% Set Proposal Title, Project # & PI Name %%%%%%
\newcommand{\proposaltitle}{}
\newcommand{\shorttitle}{}
\newcommand{\projectnum}{}
\newcommand{\piname}{}

%%%%%%%%%%%% DO NOT CHANGE the remainder of the PREAMBLE %%%%%%%%%%
%%%%%% define ``Arial'' fonts needed for headings and header %%%%%%
\newcommand{\bsarial}[1]{\fontsize{12pt}{1pt} \textbf\scshape{\textsf{#1}}}
\newcommand{\barial}[1]{\fontsize{14pt}{4pt} \bf{\scshape{\textsf{#1}}}}
\newcommand{\smarial}[1]{\fontsize{10pt}{0pt} \em{#1}}


%%%%%% set margins %%%%%%
\setlength\topmargin{-.5in}
\setlength\oddsidemargin{0.0in}
\setlength\evensidemargin{0.0in}
\setlength\textheight{9in}
\setlength\textwidth{6.5in}
%\setlength\headheight{77pt}
%\setlength\headsep{0.25in}

%%%%%% set header and footer %%%%%%
\pagestyle{fancy}
\fancyhf{}
\fancyfoot[L]{\smarial{\piname}}
\fancyfoot[R]{\smarial{Page \thepage}}
\fancyhead[L]{\smarial{\shorttitle}}
\fancyhead[R]{\smarial{\projectnum}}
\renewcommand\headrulewidth{0pt} % Removes funny header line



%%%%%%%%%% END PREAMBLE %%%%%%%%%%


\begin{document}
\begin{center}
{\large\bf \scshape Andrew Connolly}\\
\end{center}

\noindent
{\bf \scshape Education and Training}\\
\noindent
Imperial College, University of London, Ph.D. (Physics and Astronomy),  1993\\
Imperial College, University of London; Physics; B.Sc. (First Class Honors) 1988\\

\noindent
{\bf \scshape Research and Professional Experience}\\
\noindent
2011 --  present,  Professor, University of Washington, Seattle, WA

\noindent
2011 -- 2012, Visiting Faculty (Exacycle Program), Google, Seattle, PA 

\noindent
2007 --  2011,  Associate Professor, University of Washington, Seattle, WA

\noindent
2006 -- 2007, Visiting Faculty (Development of Google Sky), Google, Pittsburgh, PA 

\noindent
2004 -- 2006, Associate Professor, University of Pittsburgh, Pittsburgh, PA 

\noindent
1999 -- 2004, Assistant Professor, University of Pittsburgh, Pittsburgh, PA 


\noindent
1995 -- 1999, Associate Research Scientist, Johns Hopkins University, Baltimore, MD 


\noindent
1992 -- 1994, Postdoctoral Fellow, Johns Hopkins University, Baltimore, MD \\

\noindent
{\bf \scshape Selected Relevant Publications}\\
\noindent
{\bf[1]} vanderPlas, J., Connolly, A.J., Jain, B., and Jarvis. M., ``Interpolating Masked Weak-lensing Signal with Karhunen-Lo\`{e}ve Analysis'', 2012, ApJ, 744, 180

\noindent 
{\bf[2]} Wiley, K., Connolly, A.J., Gardner J., Krughoff, K.S., Balazinska, M., Howe, B., Kwon Y., and Bu, Y., ``Astronomy in the Cloud: Using MapReduce for Image Coaddition'', 2011, PASP, 123, 366

\noindent
{\bf[3]} Connolly, A.J., Peterson, J., Jernigan, J.G., Abel, R.,
Bankert, J., and 18 colleagues, "Simulating the LSST System",
Proceedings of SPIE Vol. 7737 (2010)
 
\noindent
{\bf[4]}  D. Suciu, A. Connolly, and B. Howe,  ``Embracing Uncertainty in Large- Scale Computational Astrophysics'', in MUD, 2009, pp. 63--77.

\noindent
{\bf[5]} Stabenau, H. F., Connolly, A., and Jain, B., "Photometric redshifts
with surface brightness priors", MNRAS, 387, 1215 (2009)

\noindent
{\bf[6]} Vanderplas, J. and Connolly, A.,  ``Reducing the Dimensionality of Data: Locally Linear Embedding of Sloan Galaxy Spectra'', 2009,  AJ,  138,  1365 

\noindent
{\bf[7]} Connolly, A., Scranton, R. and Ornduff, T., ``Google Sky: A Digital View of the Night Sky'', 2008,  ASPC,  400,  96 

\noindent
{\bf[8]} Kubica, J., Moore, A. \& Connolly, A.J., "Efficient Trajectory Based
Spatial Queries", The Eleventh ACM SIGKDD International
Conference on Knowledge Discovery and Data Mining (2005)

\noindent
{\bf[9]} Yip, C. W., Connolly, A.  J., Vanden Berk, D. E., Ma, Z., Frieman,
J. A., SubbaRao, M., Szalay, A.  S., Richards, G. T., Hall, P. B.,
Schneider, D. P., and 12 colleagues, "Spectral Classification of
Quasars in the Sloan Digital Sky Survey: Eigenspectra, Redshift, and
Luminosity Effects", AJ, 128, 2603 (2004)

\noindent
{\bf[10]} Connolly, A. J., Scranton, R., Johnston, D.,
Dodelson, S., Eisenstein, D. J., Frieman, J. A., Gunn, J. E., Hui, L.,
Jain, B., Kent, S., and 48 colleagues,  "The Angular Correlation Function of Galaxies from Early Sloan Digital Sky Survey Data", ApJ,  579,  42 (2002)


\newpage
\noindent
{\bf \scshape Synergistic Activities}\\
\noindent {\bf [1] Committees and Panels:} Image Simulation Scientist
for LSST; software and computing coordinator for Dark Energy Science
Collaboration, member of the LSST Science Council; board member of Pacific
Northwest Gigapop; board member of Astronomy and Computing journal, 
member of Physics advisory panel for arXiv; member
of NSF and NASA review and senior review panels; science organizing
committee for LSST@Europe 2013, AstroViz 2011, Astroinformatics 2010

\noindent {\bf [2] Outreach} Led the development of Sky in Google
Earth (aka ``Google Sky'') while on sabbatical at Google; a framework
for exploring the sky. Google Sky was one of the most successful
releases of software in the history of Google. Developed an affordable
planetarium system with Microsoft and World Wide Telescope.

\noindent {\bf [3] Community Support} Developer of fast algorithms for
data mining of large astronomical data sets. Software for the
implementation of these algorithms are made available through the INCA
collaboration. Co-author of a book on data analytics for astronomy
(which includes Python software for the analysis of large and complex
data sets) ``Statistics, Data Mining and Machine Learning in
Astronomy: A Practical Python Guide for the Analysis of Survey Data'',
Princeton University Press, to appear in 2013.

\noindent 
{\bf [4] Teaching } PhD advisor for ten graduate students, research
adviser for 14 undergraduate students including students supported
under an NSF funded REU program for under-represented groups in
astrophysics. Co-creator and co-lecturer for a joint Pitt/CMU
Ph.D. course for students from Computer Science, Statistics, Physics
and Biology: ``Computational Statistics of Multidimensional Scientific
Databases''.\\



\noindent
{\bf \scshape Collaborators \& Co-Editors} \\
\noindent
Tim
Axelrod (U of Arizona), Tamas Budavari
(JHU), Josh Frieman (U of Chicago), Jeff
Gardner (U Washington), Chris Genovese (Carnegie Mellon), Hans F. Stabenau (U Pennsylvania), Andrew Hopkins (U of
Sydney), Bhuvnesh Jain (U Pennsylvania), Robert Jedicke (University of
Hawaii), Simon Krughoff (U Washington), Jeremy Kubica (Google), Robert
Lupton (Princeton University), Andrew Moore (Google), Bob Nichol (U
Portsmouth), Gordon Richards (Drexel
University), Ryan Scranton (Google), Sam Schmidt (UC Davis), Jeff
Schneider (Carnegie Mellon), Ravi Sheth (U Pennsylvania), Michael Schneider (UC Davis),
Mark Subbarao (U of Chicago), Alex Szalay (JHU), Istvan
Szapudi (University of Hawaii),
Tony Tyson (UC Davis), Larry Wasserman (Carnegie Mellon), Niraj
Welikala (U of Pittsburgh), Ching-Wa Yip (JHU)\\

\noindent
{\bf \scshape Graduate and Postdoctoral Advisors and Advisees\&} \\
\noindent
{\bf [1] Graduate and Postdoctoral Advisors}
\noindent
Alex Szalay (Johns
Hopkins), Bob Joseph (U. of Hawaii)

\noindent
{\bf [2] Postdoctoral Advisees}
Joerg Colberg
(Carnegie Mellon), Alberto Conti
(STScI), Scott Daniel (UW), Jeff Gardner (UW), Rob Gibson (UW), Andrew
Hopkins (U of Sydney), Diego Marcos (UW), Simon Krughoff (UW), James Pizagno (UW), Ryan Scranton (Davis),  Nicole Silvestre (UW), Dan vanden Berk (Penn State), Keith Wiley (UW)


\noindent
{\bf [3] Graduate Student Advisees}
 Yusra AlSayyad (UW), Robert
Brunner (U Illinois), Jeremy Brewer (U of
Pittsburgh), Tamas Budavari (JHU), Cameron McBribe (U of Pittsburgh), Sam Schmidt
(UC Davis), Gyula Szokoly (Max-Planck fur
Extraterrestrische Physik), Jake VanderPlas (UW), 
Niraj Welikala (U of Pittsburgh), Ching-Wa Yip (JHU).

\end{document}


\section{RESEARCH PROGRAM}
\label{sec:research}

The research plan under this grant will have as its primary goal to
introduce and prepare students to perform research in computationally
intense statistical modeling and in the statistical analysis of large
scientific data. The students targeted will be Statistics and ACMS
Statistics track undergraduates, as well as Statistics and Astronomy
graduate students. 

We will specifically foster working in teams of graduate and
undergraduate students. The students will be mentored by the four
PI's. Some potential research projects are listed below.
\bits
\item {\em Non-linear dimension reduction in large scientific data} is a long-time research interest of \meila, with methodological results in \cite{PerraultM:asymptotic-nips11,Perrault-JoncasM:riemann-jmlr11}. Currently, these methods have been used largely for data visualisation in Astronomy as well as in other scientific fields; however, we believe they could be useful beyond visualisation, in discovering low dimensional meaningful parametrizations of high-dimensional data. The students will work on implementing scalable Python versions of the algorithms, suitable for large data sets, and, co-advised by the PI's will apply them to the study of scientific data.
  %% Should we mention Vanderplas & Connolly LLE work here?
\item {\em Non-parametric clustering and density estimation.} We will
  focus here on methods that cluster by finding high-density regions
  in very large data \cite{cheng:95,nugent:10,rinaldo:13}. This method
  is particularly suited to finding non-spherical (or non-ellipsoidal)
  clusters in low-dimensional data. It is robust to the presence of
  outliers or of background distributions of points, and it is also
  robust to the presence of clusters of very different sizes, being
  able to accomodate together in one model both very large and very
  small clusters. The algorithms are easy to understand and implement,
  making them ideal for a first project in big scientific data
  analysis. Moreover, with an appropriate distance function, these
  methods can be applied to very high dimensional data as well. Our
  teams will explore finding appropriate distance functions in
  recorded brain activity data, social network data, and astronomical
  data.
\item {\em Classification of high-dimensional data sets.} 
Both supervised and unsupervised classification methods have a long
history in astronomy. We will use large astronomical data sets (e.g. 
stellar and quasar samples from the SDSS, with millions of objects in
each class), with emphasis on most recent surveys, to expose students
to a suite of modern classification methods. We will emphasise critical
comparison of different methods and how to choose an optimal method
for the specific domain problems at hand.
\item {\em Time series analysis.} 
We will focus here on methods for finding periodicity in time series 
data. Starting from standard methods which essentially fit a single 
harmonic model to data with Gaussian noise, we will explore how
to incorporate non-sinusoidal periodicity (perhaps using template 
light curves) and non-gaussian noise. We will also explore methods 
for analysis of stochastic data, such as wavelet analysis and Gaussian
processes with arbitrary covariance.
\eits
%
Results of this work may be part of publishable research; others will
be converted into problems and data sets for the teaching
infrastructure that we will be building. We expect (see also Section
\ref{sec:activities-grad}) that graduate students funded by this grant
will transition after their ``apprenticeship'', to other big data
research projects funded by other sources. 
